% Options for packages loaded elsewhere
\PassOptionsToPackage{unicode}{hyperref}
\PassOptionsToPackage{hyphens}{url}
%
\documentclass[
  man,floatsintext]{apa7}
\usepackage{amsmath,amssymb}
\usepackage{iftex}
\ifPDFTeX
  \usepackage[T1]{fontenc}
  \usepackage[utf8]{inputenc}
  \usepackage{textcomp} % provide euro and other symbols
\else % if luatex or xetex
  \usepackage{unicode-math} % this also loads fontspec
  \defaultfontfeatures{Scale=MatchLowercase}
  \defaultfontfeatures[\rmfamily]{Ligatures=TeX,Scale=1}
\fi
\usepackage{lmodern}
\ifPDFTeX\else
  % xetex/luatex font selection
\fi
% Use upquote if available, for straight quotes in verbatim environments
\IfFileExists{upquote.sty}{\usepackage{upquote}}{}
\IfFileExists{microtype.sty}{% use microtype if available
  \usepackage[]{microtype}
  \UseMicrotypeSet[protrusion]{basicmath} % disable protrusion for tt fonts
}{}
\makeatletter
\@ifundefined{KOMAClassName}{% if non-KOMA class
  \IfFileExists{parskip.sty}{%
    \usepackage{parskip}
  }{% else
    \setlength{\parindent}{0pt}
    \setlength{\parskip}{6pt plus 2pt minus 1pt}}
}{% if KOMA class
  \KOMAoptions{parskip=half}}
\makeatother
\usepackage{xcolor}
\usepackage{longtable,booktabs,array}
\usepackage{calc} % for calculating minipage widths
% Correct order of tables after \paragraph or \subparagraph
\usepackage{etoolbox}
\makeatletter
\patchcmd\longtable{\par}{\if@noskipsec\mbox{}\fi\par}{}{}
\makeatother
% Allow footnotes in longtable head/foot
\IfFileExists{footnotehyper.sty}{\usepackage{footnotehyper}}{\usepackage{footnote}}
\makesavenoteenv{longtable}
\usepackage{graphicx}
\makeatletter
\def\maxwidth{\ifdim\Gin@nat@width>\linewidth\linewidth\else\Gin@nat@width\fi}
\def\maxheight{\ifdim\Gin@nat@height>\textheight\textheight\else\Gin@nat@height\fi}
\makeatother
% Scale images if necessary, so that they will not overflow the page
% margins by default, and it is still possible to overwrite the defaults
% using explicit options in \includegraphics[width, height, ...]{}
\setkeys{Gin}{width=\maxwidth,height=\maxheight,keepaspectratio}
% Set default figure placement to htbp
\makeatletter
\def\fps@figure{htbp}
\makeatother
\setlength{\emergencystretch}{3em} % prevent overfull lines
\providecommand{\tightlist}{%
  \setlength{\itemsep}{0pt}\setlength{\parskip}{0pt}}
\setcounter{secnumdepth}{-\maxdimen} % remove section numbering
% Make \paragraph and \subparagraph free-standing
\ifx\paragraph\undefined\else
  \let\oldparagraph\paragraph
  \renewcommand{\paragraph}[1]{\oldparagraph{#1}\mbox{}}
\fi
\ifx\subparagraph\undefined\else
  \let\oldsubparagraph\subparagraph
  \renewcommand{\subparagraph}[1]{\oldsubparagraph{#1}\mbox{}}
\fi
\newlength{\cslhangindent}
\setlength{\cslhangindent}{1.5em}
\newlength{\csllabelwidth}
\setlength{\csllabelwidth}{3em}
\newlength{\cslentryspacingunit} % times entry-spacing
\setlength{\cslentryspacingunit}{\parskip}
\newenvironment{CSLReferences}[2] % #1 hanging-ident, #2 entry spacing
 {% don't indent paragraphs
  \setlength{\parindent}{0pt}
  % turn on hanging indent if param 1 is 1
  \ifodd #1
  \let\oldpar\par
  \def\par{\hangindent=\cslhangindent\oldpar}
  \fi
  % set entry spacing
  \setlength{\parskip}{#2\cslentryspacingunit}
 }%
 {}
\usepackage{calc}
\newcommand{\CSLBlock}[1]{#1\hfill\break}
\newcommand{\CSLLeftMargin}[1]{\parbox[t]{\csllabelwidth}{#1}}
\newcommand{\CSLRightInline}[1]{\parbox[t]{\linewidth - \csllabelwidth}{#1}\break}
\newcommand{\CSLIndent}[1]{\hspace{\cslhangindent}#1}
\ifLuaTeX
\usepackage[bidi=basic]{babel}
\else
\usepackage[bidi=default]{babel}
\fi
\babelprovide[main,import]{english}
% get rid of language-specific shorthands (see #6817):
\let\LanguageShortHands\languageshorthands
\def\languageshorthands#1{}
% Manuscript styling
\usepackage{upgreek}
\captionsetup{font=singlespacing,justification=justified}

% Table formatting
\usepackage{longtable}
\usepackage{lscape}
% \usepackage[counterclockwise]{rotating}   % Landscape page setup for large tables
\usepackage{multirow}		% Table styling
\usepackage{tabularx}		% Control Column width
\usepackage[flushleft]{threeparttable}	% Allows for three part tables with a specified notes section
\usepackage{threeparttablex}            % Lets threeparttable work with longtable

% Create new environments so endfloat can handle them
% \newenvironment{ltable}
%   {\begin{landscape}\centering\begin{threeparttable}}
%   {\end{threeparttable}\end{landscape}}
\newenvironment{lltable}{\begin{landscape}\centering\begin{ThreePartTable}}{\end{ThreePartTable}\end{landscape}}

% Enables adjusting longtable caption width to table width
% Solution found at http://golatex.de/longtable-mit-caption-so-breit-wie-die-tabelle-t15767.html
\makeatletter
\newcommand\LastLTentrywidth{1em}
\newlength\longtablewidth
\setlength{\longtablewidth}{1in}
\newcommand{\getlongtablewidth}{\begingroup \ifcsname LT@\roman{LT@tables}\endcsname \global\longtablewidth=0pt \renewcommand{\LT@entry}[2]{\global\advance\longtablewidth by ##2\relax\gdef\LastLTentrywidth{##2}}\@nameuse{LT@\roman{LT@tables}} \fi \endgroup}

% \setlength{\parindent}{0.5in}
% \setlength{\parskip}{0pt plus 0pt minus 0pt}

% Overwrite redefinition of paragraph and subparagraph by the default LaTeX template
% See https://github.com/crsh/papaja/issues/292
\makeatletter
\renewcommand{\paragraph}{\@startsection{paragraph}{4}{\parindent}%
  {0\baselineskip \@plus 0.2ex \@minus 0.2ex}%
  {-1em}%
  {\normalfont\normalsize\bfseries\itshape\typesectitle}}

\renewcommand{\subparagraph}[1]{\@startsection{subparagraph}{5}{1em}%
  {0\baselineskip \@plus 0.2ex \@minus 0.2ex}%
  {-\z@\relax}%
  {\normalfont\normalsize\itshape\hspace{\parindent}{#1}\textit{\addperi}}{\relax}}
\makeatother

\makeatletter
\usepackage{etoolbox}
\patchcmd{\maketitle}
  {\section{\normalfont\normalsize\abstractname}}
  {\section*{\normalfont\normalsize\abstractname}}
  {}{\typeout{Failed to patch abstract.}}
\patchcmd{\maketitle}
  {\section{\protect\normalfont{\@title}}}
  {\section*{\protect\normalfont{\@title}}}
  {}{\typeout{Failed to patch title.}}
\makeatother

\usepackage{xpatch}
\makeatletter
\xapptocmd\appendix
  {\xapptocmd\section
    {\addcontentsline{toc}{section}{\appendixname\ifoneappendix\else~\theappendix\fi\\: #1}}
    {}{\InnerPatchFailed}%
  }
{}{\PatchFailed}
\keywords{Core threats, Motivation, Measurement}
\usepackage{lineno}

\linenumbers
\usepackage{csquotes}
\makeatletter
\renewcommand{\paragraph}{\@startsection{paragraph}{4}{\parindent}%
  {0\baselineskip \@plus 0.2ex \@minus 0.2ex}%
  {-1em}%
  {\normalfont\normalsize\bfseries\typesectitle}}

\renewcommand{\subparagraph}[1]{\@startsection{subparagraph}{5}{1em}%
  {0\baselineskip \@plus 0.2ex \@minus 0.2ex}%
  {-\z@\relax}%
  {\normalfont\normalsize\bfseries\itshape\hspace{\parindent}{#1}\textit{\addperi}}{\relax}}
\makeatother

\ifLuaTeX
  \usepackage{selnolig}  % disable illegal ligatures
\fi
\IfFileExists{bookmark.sty}{\usepackage{bookmark}}{\usepackage{hyperref}}
\IfFileExists{xurl.sty}{\usepackage{xurl}}{} % add URL line breaks if available
\urlstyle{same}
\hypersetup{
  pdftitle={The Core Threat Structured Interview},
  pdfauthor={Elad Zlotnick1 \& Jonathan D. Huppert1},
  pdflang={en-EN},
  pdfkeywords={Core threats, Motivation, Measurement},
  hidelinks,
  pdfcreator={LaTeX via pandoc}}

\title{The Core Threat Structured Interview}
\author{Elad Zlotnick\textsuperscript{1} \& Jonathan D. Huppert\textsuperscript{1}}
\date{}


\shorttitle{CTSI}

\authornote{

This work was supported by ISF 1905/20 awarded to Jonathan Huppert, Sam and Helen Beber Chair of Clinical Psychology.

The authors made the following contributions. Elad Zlotnick: Conceptualization, Methodology, Formal Analysis, Writing - Original Draft Preparation, Writing - Review \& Editing; Jonathan D. Huppert: Writing - Review \& Editing, Supervision.

Correspondence concerning this article should be addressed to Elad Zlotnick, Department of Psychology, The Hebrew University of Jerusalem, Mount Scopus, Jerusalem 91905, Israel. E-mail: \href{mailto:elad.zlotnick@mail.huji.ac.il}{\nolinkurl{elad.zlotnick@mail.huji.ac.il}}

}

\affiliation{\vspace{0.5cm}\textsuperscript{1} The Hebrew University of Jerusalem}

\abstract{%
Core threats, the ultimate feared consequences underlying avoidant behaviors, often underlie pathological anxiety.
For example, contamination fears can be driven by core threats such as death, harming loved ones, or disgust.
Despite their clinical importance, core threats are under-researched.
This study is the first systematic examination of core threats, via a standardized semi-structured interview, the Core Threat Structured Interview (CTSI).
Core threats were examined via face-to-face and online formats in four samples.
Across studies, the CTSI demonstrated robust reliability (e.g., interrater, test-retest) and validity (convergent, divergent).
In addition, a reliable coding scheme was used to categorize core threats.
Results revealed that core threat themes were distinct from proximal threats, and were stable.
Different obsessions in the same individual sometimes had shared core threats.
By enabling individualized assessments, we were able to provide a nuanced approach to understanding anxiety, paving the way for research into the motivational mechanisms driving fear.
}



\begin{document}
\maketitle

\begin{quote}
Happy families are all alike; every unhappy family is unhappy in its own way. \emph{Leo Tolstoy, Anna Karenina}
\end{quote}

A patient walks into the clinic.
You can immediately see her careful demeanor and chafed hands.
She reports that she compulsively washes her hands, that she doesn't go to public restrooms, and that she carefully cleans doorknobs before touching them.
What is she afraid of?
When asked, she says that she is afraid of being contaminated.
But is contamination the true threat underlying her fear?
Initial reports of a patient's fears often mask deeper concerns (Borkovec et al., 1998).
This patient might fear becoming sick and dying, contaminating her loved ones, being deemed disgusting and rejected, or simply suffering.
The key is that we can't know the true threat until we ask.

These underlying concerns, known as core threats, are also referred to as core fears, catastrophic beliefs, or central innately aversive outcomes.
Core threats are central to treating anxiety disorders (Huppert \& Zlotnick, 2012; Zlotnick \& Huppert, 2025).
Many clinicians recognize the importance of understanding core threats to inform effective interventions
(e.g., Craske et al., 2022; Gillihan et al., 2012; Murray, Loeb, et al., 2016; Pinciotti et al., 2021).
However, despite their significance, clinical guidelines for identifying and addressing core threats remain limited.
This paper aims to fill this gap by offering practical guidelines and exploring the phenomenology of core threats.

\subsection{What are Core Threats?}\label{what-are-core-threats}

Fear and anxiety are adaptive responses to perceived threats.
The nature of these threats often follows a hierarchical pattern.
Consider a jungle, where dangers such as venomous snakes, prowling lions, or quicksand abound.
In this context, the jungle represents the proximal threat, the immediate signal of danger.
However, the core threat---the ultimate feared outcome---is death (for example).
It is the plausibility of death that makes the jungle threatening and evokes fear.

Core threats arise from the interplay between what individuals \emph{expect} (e.g., the likelihood of death) and their evaluation, or the \emph{meaning} that they assign (e.g., the significance or consequences of potential death).
This framework helps explain why the same situation may evoke different core threats for different people.
For instance, individuals might \emph{expect} different dangers but assign a similar \emph{meaning}:
one person in the jungle may \emph{expect} to encounter a snake, while another \emph{expects} a lion, yet both perceive the ultimate threat (meaning) as death.
Conversely, individuals might \emph{expect} the same danger but assign different \emph{meanings}:
for someone \emph{expecting} an encounter with snakes, the \emph{meaning} might center on their unpredictability, for another on their sliminess, for a third on the immediate risk to their life, and for yet another on the broader impact of their potential death on loved ones.

These distinctions are critical when examining anxiety disorders (e.g., Gillihan et al., 2012; Murray, Loeb, et al., 2016; Pinciotti et al., 2021).
In pathological anxiety, seemingly benign stimuli are perceived as dangerous.
To determine their safety, it is essential to understand the specific nature of the threat attributed to them (Craske et al., 2022; Gillihan et al., 2012; Huppert \& Zlotnick, 2012; Murray, Loeb, et al., 2016).
Consider an individual who fears blood.
If their core threat involves the stress of encountering blood-like stimuli, exposure to sheep blood might be effective.
However, if their primary fear centers on contracting AIDS, such exposure would likely be ineffective.
The same principle applies to thought challenges and behavioral experiments.

A significant challenge in safety learning is its limited generalization across contexts (see Bouton, 2002).
By focusing on core threats, clinicians can identify the most threatening aspects of feared stimuli, thereby promoting better generalization of safety learning across contexts (Gillihan et al., 2012; Zlotnick \& Huppert, 2025).
Identifying core threats also enhances clinicians' understanding of patients' experiences when confronting their fears.
This understanding fosters a sense of being supported for the patient and provides a coherent narrative to explain their pathological behavior.
Consequently, determining core threats can significantly shape the trajectory of psychotherapy---from the initial case formulation (Persons, 2012) to the implementation of specific interventions like exposures, thought challenges, or behavioral experiments.

\subsection{Determining Core Threats}\label{determining-core-threats}

Accurately identifying core threats is a complex process.
It requires a clear understanding of what core threats are and the types of questions best suited to uncover them.
A semi-structured interview can be a useful tool for this purpose (Samuel et al., 2020), offering benefits for both clinical and research applications.
In clinical settings, such interviews help therapists identify the specific motivations underlying anxiety-related behaviors.
This, in turn, facilitates the development of a clear case formulation (Persons, 2012) and enables the creation of tailored interventions (e.g., Gillihan et al., 2012; Murray, Loeb, et al., 2016; Pinciotti et al., 2021).
In research, semi-structured interviews ensure consistency and accuracy, reducing ambiguity in identifying core threats and enhancing the reliability and validity of findings.
Moreover, these interviews allow the assignment of probability and threat values to both proximal and core threats.
Tracking these values across treatment can provide valuable insights into therapeutic change and its relationship to other constructs.

\subsubsection{The Catastrophizing Interview}\label{the-catastrophizing-interview}

The catastrophizing interview is a well-established procedure for investigating catastrophizing in Generalized Anxiety Disorder (GAD) and related disorders (Davey, 2006; Vasey \& Borkovec, 1992).
Developed by Vasey and Borkovec (1992), the procedure is based on the decatastrophizing technique used in cognitive therapy (Kendall \& Ingram, 1987).
The interview consists of two phases: topic generation and catastrophizing.
During topic generation, participants list their current worries, rate the percentage of time spent worrying about each topic, and evaluate its significance.
The topic with the highest percentage is then selected for the catastrophizing phase.
Participants are asked, ``What is it about {[}selected worry topic{]} that worries you?'' and then, ``What about {[}participant's response{]} would you find fearful or bad if it did actually happen?'' This questioning continues until participants either refuse to continue, cannot generate further responses, or repeat the same response three times.

The procedure was later refined to improve standardization (Davey, 2006).
Participants were instructed to write concise, single-sentence responses for each step on a response sheet.
Examples of typical catastrophizing steps were provided beforehand to familiarize participants with the process.
These updates reduced variability in responses and enhanced accessibility.
Initially designed for GAD, the procedure was later adapted for worry in insomnia (Harvey \& Greenall, 2003) and rumination in depression (Watkins \& Mason, 2002).
It primarily assesses the tendency to perseverate in worry by quantifying the number of catastrophizing steps.

While effective for measuring perseverative worry, the catastrophizing interview is not designed to identify the underlying threat that triggers fear.
Investigating core threats requires a distinct approach focused on uncovering the ultimate fear driving anxiety.
To address these challenges, we developed a tailored interview specifically for identifying core threats.

\subsubsection{The Core Threat Structured Interview}\label{the-core-threat-structured-interview}

The Core Threat Structured Interview (CTSI) begins by identifying a focal proximal threat\footnote{The CTSI manual can be found in the supplementary materials.}.
This involves identifying situations or stimuli that induce fear or are avoided, and any rituals or safety behaviors the individual engages in.
Once a set of fear responses is identified, participants select the situation that causes the most distress or negative impact on their lives.

To identify core threats, the CTSI adapts the classic ``downward arrow'' technique (e.g., Dugas \& Koerner, 2005).
Unlike the traditional focus on chains of beliefs (J. S. Beck, 2011), the CTSI emphasizes events, guiding participants through the question, ``And then what would happen?'' (Huppert \& Zlotnick, 2012).
Participants are first asked what they fear will happen if they refrain from avoidance or safety behaviors related to their chosen proximal threat.
Follow-up questions such as ``And then what?'', ``What is so terrible about that?'', and ``What does that mean to you?'' are used to explore progressively deeper fears (J. S. Beck, 2011; Leahy, 2003).
Through this iterative process, the interview continues until the underlying core threat is identified.

Core threats consist of an \emph{expectation} that an event could occur and an \emph{evaluation} that the event would be catastrophic (Zlotnick \& Huppert, 2025).
These roughly translate to likelihood and cost (e.g., Foa \& Kozak, 1986).
The evaluation depends on an individual's unique values, goals, and motivations.
Therefore, it is helpful to ask not only what might happen but also what the event would \emph{mean} to them or why it matters so much.
This approach often leads in surprising directions (i.e., a form of guided discovery; Padesky, 1993).
For instance, one woman worried that her children were abusing drugs.
When asked what was so horrible about that, she explained it meant her children were not sharing everything with her, which in turn signified to her that she was failing as a mother.

In practice, the link between proximal and core threats often follows a chain of progressively more threatening outcomes.
For example, an individual might state, ``If I don't wash my hands, I will be contaminated, leading to illness, which will hinder my ability to function, and ultimately sabotage my career.''
Another individual might say, ``A burglar might break into my house, harm or kidnap my child, and I couldn't bear that, as ensuring my family's safety and growth is the most crucial part of my life.''
In some cases, individuals describe multiple possible outcomes.
When this happens, they are encouraged to explore the branch they find most threatening.
The interview concludes when the participant cannot or will not identify a deeper threat, or when further questioning becomes repetitive.
Importantly, these branches are highly idiosyncratic, and generic pathways are insufficient.

At times, individuals may ``overshoot'' their core threat, describing their response to it instead of the threat itself.
For example, someone who fears their family dying in a car crash may upon further inquiry describe fear of falling into depression.
In such cases, the core threat is likely their family dying.
To clarify, the interviewer can explicitly compare the options: ``What would be worse for you: having your family die or sinking into depression?''
However, there are instances where the response is indeed the feared outcome.
For example, an individual may fear becoming so disgusted or anxious that they can no longer function, care for their family, or maintain relationships.
These nuances highlight the importance of careful exploration to accurately identify the core threat.

\subsubsection{Getting at Deeper Motivations}\label{getting-at-deeper-motivations}

Identifying core threats can be challenging, as simply asking what could happen is often insufficient.
Understanding the processes that create discrepancies between proximal and core threats can help address these difficulties.
Two key processes that interfere with identifying core threats are avoidance and difficulty accessing emotional cognitions.
These two issues will be explored in detail below.

One explanation for the difficulty to uncover core threats is avoidance.
Individuals may focus on immediate, proximal threats because confronting deeper, more global threats is distressing.
Encouraging individuals to endure this discomfort and approach their fears can often facilitate access to core threats.

Another explanation involves the nature of underlying threats, which are often evident only in emotional reasoning or ``hot'' cognitions.
These can be difficult to access in calmer, more reflective environments (David \& Szentagotai, 2006; see Safran \& Greenberg, 1982).
To address this, clinicians use techniques designed to tap into emotional reasoning.
In the CTSI, individuals are encouraged to focus on their \emph{feelings} rather than their ``cold'' cognitive appraisals.
Asking, ``What do you \emph{feel} might happen?'' instead of ``What do you \emph{think} might happen?'' highlights the distinction between emotional and cognitive reasoning, helping to uncover hidden core threats.

Another technique involves the use of imagery, which research shows evokes stronger emotional responses than verbal processing alone (Holmes \& Mathews, 2005, 2010).
By guiding individuals to imagine threatening scenarios vividly, clinicians can bring emotions and memories to the surface, providing better access to hot cognitions.
Together, these techniques offer complementary pathways for uncovering the deeper motivations behind anxiety-related behaviors.

\subsection{Hypotheses}\label{hypotheses}

The current study examines the phenomenology of core threats as measured by the CTSI.
We propose the following hypotheses:

\begin{enumerate}
\def\labelenumi{\arabic{enumi}.}
\tightlist
\item
  Core threats differ significantly from proximal threats.

  \begin{enumerate}
  \def\labelenumii{\alph{enumii}.}
  \tightlist
  \item
    Core threats cannot be reliably predicted based on proximal threats.
  \item
    Core threats exhibit greater variability than proximal threats.
  \end{enumerate}
\item
  A single core threat often underlies and motivates multiple proximal threats.
\item
  Core threats remain stable over time, demonstrating consistency across repeated assessments.
\end{enumerate}

By exploring the relationship between core and proximal threats and testing these hypotheses through the CTSI, this study aims to deepen our understanding of the fundamental processes that drive fear and anxiety disorders.
These insights may contribute to refining theoretical models and improving clinical interventions by emphasizing the role of core threats in shaping anxiety-related behaviors.

\section{Transparency and Openness}\label{transparency-and-openness}

\subsubsection{Preregistration}\label{preregistration}

The study was pre-registered on AsPredicted (\url{https://aspredicted.org/89j7-5m4t.pdf}) after data collection was completed but before coding of threat types began.
The pre-registration included hypotheses, methods, data collection procedures, and analysis plans, ensuring that the analyses were planned without influence from the raw data.

Several deviations from the pre-registered plan occurred:

\begin{enumerate}
\def\labelenumi{\arabic{enumi}.}
\tightlist
\item
  We used three judges to categorize core threats instead of two to increase reliability. All judges coded the threats simultaneously, ensuring that the addition was not done to affect initial outcomes.
\item
  Due to the complexity of the data structure, we were unable to apply information methods (Theil's U) to assess the agreement between sets of threats. Instead, we used permutation tests, as described below to investigate the same hypotheses.
\item
  The third experimental group (high anxiety, online CTSI) was reported as two separate groups---Hebrew-speaking and English-speaking---due to demographic differences.
\end{enumerate}

\subsubsection{Data, materials, code, and online resources}\label{data-materials-code-and-online-resources}

Supplementary materials, including datasets, analysis scripts, and detailed methodological documentation, are available on GitHub at \url{https://github.com/eladzlot/ctsi-2025-public}.
The datasets have been redacted to include only quantitative information to ensure participant confidentiality, as the core threats and other open-ended responses could potentially identify specific individuals.

\subsubsection{Reporting}\label{reporting}

We report how we determined all data exclusions, all manipulations, and all measures in the study.
This study involved an analysis of existing data rather than new data collection, thus we do not report how we determined the sample size for each study.

\subsubsection{Ethical approval}\label{ethical-approval}

All studies were approved by the ethical review board of the The Hebrew University of Jerusalem.

\section{Methods}\label{methods}

\subsection{Design}\label{design}

This study comprised four experiments, each involving distinct samples of participants who completed the CTSI.
Experiment 1 targeted individuals with obsessive-compulsive symptoms, who participated in face-to-face interviews conducted via Zoom.
Experiment 2 also employed Zoom interviews but focused on a transdiagnostic sample of individuals with high anxiety levels.
Experiment 3 introduced a digital, self-administered version of the CTSI, allowing participants to complete the interview independently online.
Experiment 4 replicated the methodology of Experiment 3 with an international, English-speaking sample.
Given that results are presented in a similar fashion for each experiment, and it is useful to the reader to be able to see the results comparatively across the four Experiments, we present the unique methods for each of the four experiments below, and following this, we present the results of all four experiments simultaneously.

\subsection{Measures}\label{measures}

\textbf{Core Threat Structured Interview (CTSI):} The CTSI is a semi-structured interview developed to identify the core threats driving fear or anxiety in individuals with anxiety disorders.
It includes a series of questions aimed at uncovering the central trigger of fear or anxiety, followed by a guided process (using a ``downward arrow'' technique) to reveal the underlying core threat, and concludes with a section focused on clarifying the idiographic meaning of the threat for the individual.
The CTSI was administered face-to-face via Zoom in the first two samples, targeting individuals with OCD and transdiagnostic anxiety.
For the last two samples, a self-administered online version of the interview was used.
During Experiments 1 and 2, minor adjustments were made to refine the interview process; these adjustments did not alter the fundamental structure of the CTSI or its aims.
Both the face-to-face and online versions of the CTSI are available in the supplementary materials.

\subsection{Data Analysis}\label{data-analysis}

\subsubsection{Rating Threats}\label{rating-threats}

This study addresses the extent to which different descriptions of threats are similar.
Specifically, we examine whether proximal and core threats align and whether core threats remain consistent across time.
To systematically assess this similarity, threats were categorized, enabling structured analysis.

Our investigation is primarily focused on identifying the threats driving fear responses.
These threats are manifestations of the values motivating these responses (Zlotnick \& Huppert, 2025).
While several taxonomies of basic values exist (see Austin \& Vancouver, 1996; Ryan, 2012 for reviews), none adequately cover the types of threats commonly associated with fear, including Schwartz's taxonomy (1992), which was previously suggested for organizing core threats (Huppert \& Zlotnick, 2012).
To fill this gap, we developed a novel taxonomy of global motivations based on clinical insights, expert consultations, and theoretical frameworks such as those advanced by Dweck (2017).
This taxonomy, detailed in Table \ref{tab:values}, was used to code both proximal and core threats.

Three trained research assistants applied this taxonomy to categorize threats.
Motivations underlying threats are often complex and may arise from multiple sources.
For example, fear of contamination might stem from both concerns about sickness leading to death and feelings of disgust.
Coders identified the primary motivation underlying each threat.
When two motivations were equally prominent, both were recorded.
If no clear dominant motivation emerged or the threat description lacked clarity, the threat was classified as ambiguous.
This approach accommodates variability in how threats are reported and interpreted.
Agreement between judges or across threats was defined as sharing at least one common motivation.
Detailed coding criteria and instructions are provided in the supplementary materials.

Coders were trained using external datasets not included in the current analyses.
Each judge independently coded the dataset, and discrepancies were resolved through consensus discussions.
This rigorous adjudication process ensured consistency in applying the taxonomy.

\begin{longtable}[]{@{}
  >{\raggedright\arraybackslash}p{(\columnwidth - 4\tabcolsep) * \real{0.1250}}
  >{\raggedright\arraybackslash}p{(\columnwidth - 4\tabcolsep) * \real{0.4265}}
  >{\raggedright\arraybackslash}p{(\columnwidth - 4\tabcolsep) * \real{0.4485}}@{}}
\caption{\label{tab:values} The Taxonomy of Values Used for Coding Threats.}\tabularnewline
\toprule\noalign{}
\begin{minipage}[b]{\linewidth}\raggedright
Category of Drives
\end{minipage} & \begin{minipage}[b]{\linewidth}\raggedright
Description
\end{minipage} & \begin{minipage}[b]{\linewidth}\raggedright
Examples of Core Threats
\end{minipage} \\
\midrule\noalign{}
\endfirsthead
\toprule\noalign{}
\begin{minipage}[b]{\linewidth}\raggedright
Category of Drives
\end{minipage} & \begin{minipage}[b]{\linewidth}\raggedright
Description
\end{minipage} & \begin{minipage}[b]{\linewidth}\raggedright
Examples of Core Threats
\end{minipage} \\
\midrule\noalign{}
\endhead
\bottomrule\noalign{}
\endlastfoot
Affiliation & The drive to form social or interpersonal bonds and avoid rejection. & Rejection, social isolation, loneliness, being alone, social awkwardness. This covers anything from family connections to wide social acceptance \\
Predictability & The drive to understand and make sense of one's environment. & Confusion, uncertainty, unpredictability. \\
Competence & The drive for effectiveness, capability, and practical functioning. & Failure, not knowing what to do, incompetence. \\
Agency \& Control & The drive to have agency and be in control of one's self and environment. & Losing control, doing something unintended, helplessness. \\
Survival & The goal of staying alive. & Death of oneself, or a close other (survival by proxy). \\
Physical Comfort & The goal of avoiding physical discomfort or suffering. & Pain, injury, physical harm to self or close other (physical comfort by proxy) \\
Self-Image & The desire to maintain a positive self-image and avoid a negative one. & Feeling worthless, evil, or guilty. Feeling that I've let myself down. \\
Morality & The drive to behave ethically and uphold virtues. & Harming others, violating religious beliefs or morals. \\
Distress (in)tolerance & The desire to avoid psychological distress or discomfort. & Disgust, Not-just-right experiences, pure psychic pain. \\
\end{longtable}

Krippendorff's \(\alpha\) was selected as the statistical measure of inter-rater reliability due to its robustness and flexibility.
This measure is widely used to assess consistency among raters across various data types, including nominal, ordinal, interval, and ratio scales (Hallgren, 2012; A. F. Hayes \& Krippendorff, 2007).
Its ability to accommodate more than two raters and handle missing data makes it particularly well-suited for our study, where raters categorized mixed data types.
Krippendorff's \(\alpha\) values are interpreted as follows: values between 0.60 and 0.74 indicate moderate agreement, suitable for exploratory research; values between 0.75 and 0.84 reflect good agreement; and values of 0.85 or above signify excellent reliability.
These thresholds guided our interpretation of reliability within this study.
As pre-registered, ratings of ``ambiguous'' were treated as missing data in the computation of \(\alpha\), as they do not represent a definitive decision about motivation.
This decision ensures that reliability estimates reflect only clear and specific categorizations.
Sensitivity analyses confirmed that this approach did not substantially affect the results.
Further details of the reliability scoring algorithm are provided in the supplementary materials.

\subsubsection{Agreement}\label{agreement}

This study examines the correlations between sets of threats, each consisting of one or two categories.
Traditional statistical measures, typically designed for single-category data, are insufficient to handle the complexity of multi-category sets, necessitating an alternative analytical approach.
We calculated the rate of agreement, defined as the percentage of pairs that share at least one category in common across all possible pairs to address this.
This relatively liberal criterion accounts for the inherent fuzziness in defining drives and distinguishing between the primary and secondary motivations underlying anxiety.

While Theil's U was pre-registered as the information-theoretic measure to assess agreement, we encountered difficulties applying it due to the complex structure of our data.
As a result, we switched to permutation tests, a non-parametric approach well-suited to handling this complexity (Edgington et al., 2007).
Unlike traditional methods, permutation tests make no assumptions about the underlying data distribution.
By repeatedly shuffling the data, these tests generate a null distribution of the test statistic, enabling the computation of p-values based on the proportion of permutations yielding a statistic as extreme as, or more extreme than, the observed one (Ernst, 2004).

This method is robust for small samples like ours and circumvents the limitations of standard analytical techniques, such as information-theoretic measures, which are ill-suited to our data.
We report the expected number of agreements, the empirical count of agreements, and the likelihood (p) of obtaining the empirical count by chance, assuming the expected count is accurate.
Consistent with conventional statistical practice, \(p<.05\) is considered significant.

\subsubsection{Motivational Diversity}\label{motivational-diversity}

Hypothesis 1b posits that core threats are associated with diverse motivations, whereas proximal threats are not.
To quantify this diversity, we employ a variation of Simpson's Diversity Index (\(D\); Simpson, 1949).
\(D\) measures ``unalikeability,'' or the probability that two randomly selected members of a population will have different motivations (Kader \& Perry, 2007).
It ranges from 0 to 1, where 0 indicates a completely heterogeneous population and 1 indicates a maximally homogeneous population (all members share the same motivations).
For interpretability, we classify \(D\) values below 0.2 as highly diverse, values between 0.2 and 0.4 as moderately diverse, and values above 0.4 as nondiverse.

While the standard form of \(D\) assumes discrete, mutually exclusive categories, our data allows for overlapping motivations.
For example, one threat might be driven by both affiliation and survival, while another is driven solely by affiliation.
To accommodate this, we adapted the calculation of \(D\) to account for partial overlaps between motivations.
Specifically, we define \(D = \frac{\sum \delta_{ij}}{N(N-1)}\), where: \(N\) is the total number of threats, \(\delta_{ij}\) is the agreement function, which returns 1 if the motivations of \(i\) and \(j\) overlap and 0 otherwise, \(i \neq j\) ensures that a member is never compared to itself.

A sample's diversity is considered greater than another if more than 95\% of bootstrapped \(D\) values for one sample exceed those of the other.
Bootstrapping is employed because it provides a robust, non-parametric method to assess variability and establish confidence intervals for \(D\) values, making it ideal for our data's structure and sample size.
This approach allows us to directly test whether core threats exhibit significantly greater motivational diversity than proximal threats, supporting or refuting the hypothesis.

\subsection{Methods: Experiment 1}\label{methods-experiment-1}

Experiment 1 aimed to evaluate the feasibility and potential utility of the face to face CTSI for identifying core threats among individuals exhibiting high levels of obsessive-compulsive (OC) symptoms.

\subsubsection{Participants}\label{participants}

Participants were drawn from a pre-existing database of individuals who had previously consented to participate in research and completed the OCI-R.
A research assistant contacted eligible participants and obtained informed consent.

Participants included 48 individuals, with 43 (89.58\%) female and 5 (10.42\%) male.
The mean age of participants was 24.60 years (SD = 3.20).
The majority identified as Jewish 40 (83.33\%), with others identifying as Christian 1 (2.08\%) or non-religious 5 (10.42\%).
All participants were Hebrew-speaking, with 41 (85.42\%) reporting Hebrew as their mother tongue.
Ethnicity data were not explicitly collected, which is noted as a limitation of the study.
In the Israeli context, ethnicity is often closely aligned with religion, and participants' religious identification may partially capture cultural background.

Marital status was distributed as follows: 40 (83.33\%) single, 7 (14.58\%) married, and 1 (2.08\%) divorced.
Educational attainment ranged from 2 to 4 years, with a mean of 2.70 years.
32 (66.67\%) of participants were employed.
Income was reported as a median of 5000-8000 ILS per month.

OCD symptom severity, measured using the OCI-R, had a mean score of 37.20 (SD = 11), with a median score of 38.50.
The median OCI-R score (38.50) placed the majority of participants well within the severe range of obsessive-compulsive symptoms (Abramovitch et al., 2020).
Moreover, 41 (85.42\%) participants met diagnostic criteria for obsessive-compulsive disorder (OCD) based on the DIAMOND interview.

\subsubsection{Procedure}\label{procedure}

Participants completed a series of questionnaires followed by a semi-structured interview conducted via Zoom, lasting between 45 minutes and two hours.
The interview included the OCD module of the DIAMOND and the CTSI.
The CTSI was used to identify core threats underlying two compulsions (proximal threats).
To enhance variability, the compulsions were chosen to be as dissimilar as possible (e.g., a cleaning compulsion versus a checking compulsion).
Interviewers explored participants' perceptions of what would occur if ritualistic behaviors were not performed.
Participants were compensated with approximately \$10 per hour or course credit, depending on their preference.

\subsubsection{Measures}\label{measures-1}

The face-to-face CTSI was employed to identify proximal and core threats.
While three independent judges initially scored each threat, inter-rater reliability was insufficient.
A consensus score was used for subsequent analyses to address this.
This approach informed the development of enhanced training protocols, which improved reliability in later experiments.
Despite initial limitations, the consensus scores are considered valid for the purposes of this study.

\textbf{Obsessive Compulsive Inventory-Revised (OCI-R; Foa et al., 2002):}
The OCI-R is an 18-item self-report measure assessing the distress associated with obsessive-compulsive symptoms using a Likert scale (0--4).
It has demonstrated strong psychometric properties across clinical and non-clinical populations (Foa et al., 2002; Huppert et al., 2007).
In this sample, the OCI-R exhibited high internal consistency, with an omega coefficient of 0.90, supporting its reliability as a measure of OCD symptom severity.

\textbf{Diagnostic Interview for Anxiety, Mood, and OCD and Related Neuropsychiatric Disorders (DIAMOND; Tolin et al., 2018):}
The DIAMOND is a semi-structured diagnostic interview developed to diagnose DSM-5 psychiatric disorders with robust psychometric properties.
For this study, only the OCD section was administered to confirm OCD diagnoses.
The DIAMOND has consistently demonstrated excellent inter-rater reliability, test-retest reliability, and strong validity metrics (Tolin et al., 2018).

\subsection{Methods: Experiment 2}\label{methods-experiment-2}

Experiment 2 aimed to build on the findings of Experiment 1 by evaluating a transdiagnostic population with elevated anxiety symptoms.
A central objective was to assess the test-retest and inter-rater reliability of the CTSI.

\subsubsection{Participants}\label{participants-1}

Participants were recruited from a database of individuals who had consented to research and completed the STICSA.
A research assistant contacted eligible participants and obtained informed consent.

Participants included 42 individuals, with 40 (95.24\%) female and 2 (4.76\%) male.
The mean age of participants was 27.80 years (SD = 6.20).
The majority identified as Jewish (41 (97.62\%)), with one identifying as non-religious (1 (2.38\%)).
All participants reported Hebrew as their mother tongue.
While ethnicity data were not explicitly collected, this was noted as a limitation of the study.
In the Israeli context, ethnicity is often closely aligned with religion, so participants' religious identification may partially capture their cultural background.

Marital status was distributed as follows: 34 (80.95\%) single, 7 (16.67\%) married, and 1 (2.38\%) divorced.
Educational attainment ranged from 2 to 4 years, with a mean of 2.80 years.
Income was reported as a median of 8000-15000 ILS per month.
Anxiety symptom severity, measured using the STICSA, had a mean score of 20.90 (SD = 13.40).

\subsubsection{Procedure}\label{procedure-1}

Participants completed the CTSI via Zoom in two sessions spaced approximately one to two months apart (Median = 33 days; Range: 25--49 days).
In the initial session, an interviewer guided participants to identify and explore the fear they perceived as most impactful.
The second session replicated the procedure, focusing on the same proximal threat but involving a different interviewer to assess inter-rater reliability.
Each session lasted 30 to 90 minutes.
Between the two sessions, 5 participants (11.9\%) withdrew from the study.
Participants were compensated with approximately \$10 per hour or course credit, depending on their preference.

\subsubsection{Measures}\label{measures-2}

The face-to-face CTSI was used to identify proximal and core threats, as outlined above.
Threats were scored by three judges who achieved good inter-rater reliability, with Krippendorff's \(\alpha\) = 0.84.

\textbf{The Trait Inventory for Cognitive and Somatic Anxiety (TICSA; Ree et al., 2008):}
The TICSA is a validated 21-item self-report measure designed to assess cognitive and somatic dimensions of trait anxiety.
Each item is rated on Likert scale (0-3), with higher scores reflecting greater severity of anxiety symptoms.
The TICSA has consistently demonstrated strong psychometric properties in diverse populations (Grös et al., 2007; Ree et al., 2008).
In the present study, the TICSA exhibited high internal consistency, with an omega coefficient of 0.95.

\subsection{Methods: Experiment 3}\label{methods-experiment-3}

Experiment 3 aimed to build on the findings from Experiment 2 by employing a self-administered version of the CTSI, facilitating a more scalable and participant-directed assessment of core and proximal threats.

\subsubsection{Participants and Procedure}\label{participants-and-procedure}

Participants were recruited from a database of individuals who had consented to research and completed the TICSA.
Each participant was contacted by a research assistant, who obtained informed consent before enrollment.
A total of 81 participants completed the study online.
TICSA symptom severity for the online sample had a mean score of 29.7 (11.3).
Due to a technical error, participant demographic data were not recorded.
However, given the similarity in recruitment procedures, the sample demographics are presumed to align closely with those of Experiment 2.
Participants were provided with a secure link to complete the self-administered CTSI along with additional relevant questionnaires.
Participants were compensated with approximately \$10 per hour or course credit, depending on their preference.

\subsubsection{Measures}\label{measures-3}

The self-administered CTSI was used to identify proximal and core threats, as discussed above.
Three judges scored each threat and achieved good inter-rater reliability (Krippendorff's \(\alpha\) = 0.86).
The TICSA was used to measure anxiety, showing high internal consistency, with an omega coefficient of 0.91.

\subsection{Experiment 4}\label{experiment-4}

Experiment 4 aimed to expand upon Experiment 3 by recruiting an international, English-speaking population.

\subsubsection{Participants and Procedure}\label{participants-and-procedure-1}

Participants were recruited via the Prolific platform for online research and compensated at a rate of £9 per hour.
Screening focused on identifying individuals with high anxiety, defined as a score greater than 4 on the Overall Anxiety Severity and Impairment Scale (OASIS; Norman et al., 2006), and functional impairment, indicated by at least one item scored above 2 on the Work and Social Adjustment Scale (WSAS; Mundt et al., 2002).
Exclusion criteria included severe depression (PHQ score \textgreater{} 14; Kroenke et al., 2009), significant post-traumatic symptoms (short PCL-5 score \textgreater{} 6; Zuromski et al., 2019), or psychotic symptoms (items 19 or 20 on the DIAMOND screener; Tolin et al., 2018).
Additional criteria included fluency in English, no history of head injury or reading/writing difficulties (as indicated on the Prolific system), and experience on the platform with an approval rate above 95\% and at least 300 prior submissions.

A total of 87 participants completed the study, with a mean age of 40.60 years (SD = 14.60).
The sample was composed of 60 (68.97\%) female and 27 (31.03\%) male participants.
The majority of participants identified their ethnicity as White 72 (82.76\%), followed by Asian 9 (10.34\%) and Black 6 (6.90\%).
Regarding educational and employment status, 14 (16.09\%) of participants were students.
Among the participants, 35 (40.23\%) were employed full-time, 21 (24.14\%) were employed part-time, and 5 (5.75\%) reported being unemployed and seeking work.

STICSA symptom severity in this sample had a mean score of 7.80 (SD = 2.90).
Eligible participants signed a consent form, completed a set of questionnaires related to a separate study, and then proceeded to the main experiment, which involved completing the self-administered CTSI.

\subsubsection{Measures}\label{measures-4}

The self-administered CTSI was used to identify proximal and core threats, as discussed above.
Three judges scored each threat, achieving good inter-rater reliability (Krippendorff's \(\alpha\) = 0.80).

\textbf{The Overall Anxiety Severity and Impairment Scale (OASIS; Norman et al., 2006):} was used to measure anxiety.
This 5-item scale assesses the frequency, intensity, and impact of anxiety and fear over the past week, with responses ranging from 0 (Little or None) to 4 (Extreme or All the Time).
Higher scores reflect greater severity, with a cut-off score of eight recommended for identifying anxiety disorders and a change of four points considered clinically significant (Moore et al., 2015).
The OASIS demonstrated excellent reliability in this study, with an omega coefficient of 0.82, consistent with its strong psychometric properties reported in previous research (Norman et al., 2006).

\section{Results}\label{results}

\subsection{The Distribution of Threat Values}\label{the-distribution-of-threat-values}

The distribution of values in core and proximal threats varied across experiments, with notable differences between the face-to-face and self-administered formats (see Figure \ref{fig:distribution}).
Proximal threats were more frequently rated as ambiguous compared to core threats.
This discrepancy may stem from the CTSI's emphasis on exploring values, though participants were not explicitly instructed to frame their threats in these terms.
A more plausible explanation is that the CTSI facilitates a clearer focus on core threats during the interview process.



\begin{figure}
\includegraphics[width=0.9\linewidth]{/home/eladzlot/projects/ctsi-2025-public/docs/output/ctsi_files/figure-latex/distribution-1} \caption{Distribution of Threat Values Across Experiments}\label{fig:distribution}
\end{figure}

\subsection{Hypothesis 1: Proximal-Core Agreement}\label{hypothesis-1-proximal-core-agreement}

Hypothesis 1 was that core threats represent a distinct psychological process from proximal threats.
To test this, we examined two key questions: whether core threats can be predicted based on proximal threats and whether core threats display greater diversity in content compared to proximal threats.
Across all experiments, core threats matched their corresponding proximal threats in no more than 32.5\% of cases, with the lowest agreement observed in Experiment 1 (High OC sample), where the rate was 16.9\%.
Among the transdiagnostic samples, agreement rates were slightly above chance.
However, after applying the Holm-Bonferroni correction for multiple comparisons (Holm, 1979), only Experiment 2 (face-to-face transdiagnostic sample) demonstrated a statistically significant agreement between proximal and core threats.
These results indicate that while proximal threats may offer some information about core threats, the observed agreement rates, are clinically insufficient.
This finding supports the hypothesis that proximal threats do not reliably predict core threats, as more than 70\% of core threats differed from their proximal counterparts.

The diversity of proximal threats, measured using Simpson's \(D\), ranged from 0.30 to 0.46 across experiments.
Core threats generally exhibited greater diversity than proximal threats, with the exception of Experiment 2 (face-to-face high anxiety sample), where diversity levels were equivalent.
The difference was most pronounced in Experiment 1 and 4 (face-to-face high OC, and online English speaking).
However, after correcting for multiple comparisons, only the diversity difference observed in Experiment 1 remained statistically significant.
This analysis was not pre-registered and should be considered exploratory.
The diversity metrics and detailed statistical results are presented in Table \ref{tab:agreement}.

\begin{table}[tbp]

\begin{center}
\begin{threeparttable}

\caption{\label{tab:agreement}Agreement and Diversity Statistics Across Studies}

\begin{tabular}{lllll}
\toprule
 & \multicolumn{2}{c}{Interview} & \multicolumn{2}{c}{Self-administered} \\
\cmidrule(r){2-3} \cmidrule(r){4-5}
 & High OC FTF & TD FTF & TD Online (He) & TD Online (En)\\
\midrule
Agreement &  &  &  & \\
\ \ \ N pairs & 71 & 78 & 77 & 84\\
\ \ \ Expected Agreement (\%) & 14 (19.72\%) & 16 (20.51\%) & 18 (23.38\%) & 22 (26.19\%)\\
\ \ \ Actual Agreement (\%) & 12 (16.90\%) & 23 (29.49\%) & 25 (32.47\%) & 24 (28.57\%)\\
\ \ \ p Agreement & .759 & .031 & .049 & .302\\
Diversity (D) &  &  &  & \\
\ \ \ Proximal threats & 0.31 [0.25, 0.38] & 0.30 [0.23, 0.38] & 0.34 [0.27, 0.45] & 0.46 [0.36, 0.58]\\
\ \ \ Core threats & 0.24 [0.20, 0.29] & 0.31 [0.25, 0.38] & 0.31 [0.25, 0.40] & 0.33 [0.26, 0.42]\\
\ \ \ p Diversity & .038 & .584 & .294 & .040\\
\bottomrule
\addlinespace
\end{tabular}

\begin{tablenotes}[para]
\normalsize{\textit{Note.} Agreement reflects the count of expected and actual pairs of motivations aligning (p Agreement indicates the likelihood of observing this agreement by chance). Diversity is measured using the Simpson Diversity Index for proximal versus core threats (p Diversity represents the probability that core threat diversity exceeds proximal threat diversity).OC - obsessive compulsive, FTF - face to face, TD - Transdiagnostic, He - Hebrew, En - English}
\end{tablenotes}

\end{threeparttable}
\end{center}

\end{table}

After identifying core threats, participants were asked whether these threats reflected their underlying motivation for fear.
This assessment was conducted in the three trans-diagnostic samples but not in the OCD sample.
The full results are presented in Figure \ref{fig:subjective}.
Across all experiments, the majority of participants affirmed that the identified core threats accurately represented their motivation for anxiety.
Notably, the proportion of participants endorsing the core threat as their ``true'' motivation was consistent regardless of whether the proximal and core threats aligned or the identified core threat reflected a different motivation.
A significant proportion of participants expressed uncertainty about whether the identified core threat truly captured their underlying motivation.
However, this uncertainty was less prevalent in the self-administered versions of the CTSI, suggesting that the self-administered format may facilitate more confidence in the identification of core threats.
This analysis was not pre-registered and should be considered exploratory.
While the results are intriguing and suggest potential avenues for further investigation, they require replication and confirmation in future research.



\begin{figure}
\centering
\includegraphics{/home/eladzlot/projects/ctsi-2025-public/docs/output/ctsi_files/figure-latex/subjective-1.pdf}
\caption{\label{fig:subjective}The mosaic plot illustrates participants' responses to the question: ``Does the core threat you identified reflect your true motivation?'' Tile sizes represent the relative frequencies of responses across groups. The plot indicates that core threats generally align with participants' true motivations. Sensitivity analyses confirmed that these proportions remained consistent regardless of whether core threats matched proximal threats.}
\end{figure}

\subsection{Hypothesis 2: Multiple Proximal Threats and One Core Threat}\label{hypothesis-2-multiple-proximal-threats-and-one-core-threat}

Hypothesis 2 was that one core threat often motivates multiple proximal threats.
We investigated this in Experiment 1 by administering the CTSI for two distinct compulsions (proximal threats).
Only 13 (27.08\%) individuals identified core threats for both compulsions.
This limited identification often occurred because participants became fatigued and impatient by the time they were interviewed about the second compulsion.
Consequently, our data on this topic is both limited and potentially biased (e.g., is there a relationship between participants' persistence and the characteristics of their fears?).

A permutation test revealed that the median expected number of agreements was 4 (30.77\%).
In practice, 7 (53.8\%, \(p\) = .075) pairs of core threats agreed with one another.
While this result is not statistically significant, it suggests that core threats may motivate multiple proximal threats within the same individual.
Due to the small sample size, these findings should be interpreted cautiously.
Further research is warranted, as this estimate might be higher if more similar compulsions were selected.

Interestingly, some pairs might share common underlying threats, even when their coded values did not match.
For example, one woman was afraid of contamination and dying if she did not perform her cleaning compulsions (Survival) and also feared breaking up with her boyfriend, not having children, and being alone forever (Affiliation).
A deeper investigation might reveal that not having children held the same significance as not surviving for her.
This dataset, collected during the initial development of the CTSI, may have been affected by suboptimal administration.
Nonetheless, this finding underscores that the same core threats often appear to motivate different proximal threats.

\subsection{Hypothesis 3: Test-Retest Validity}\label{hypothesis-3-test-retest-validity}

In the high-anxiety face-to-face sample, 33 individuals (78.6\%) completed both evaluations of their core threats.
A permutation test revealed that the median expected number of agreements was 11 cases (33.3\%).
In practice, 17 pairs of core threats agreed with each other (51.5\%, \(p\) = .011).

This finding suggests significant test-retest validity, indicating that the same core threats likely motivate fear over time.
Furthermore, it supports the notion that different evaluators can reliably identify the same core threat when interviewing an individual.
However, the agreement rate is slightly lower than expected.
Future research should investigate whether core threats are less stable than predicted or if adjustments to CTSI administration can improve test-retest reliability.

\section{Discussion}\label{discussion}

The current study aimed to provide clinical guidelines for identifying core threats in anxiety disorders and to explore the phenomenology of these threats.
Core threats, the underlying fears driving proximal threats, play a crucial role in the structure and treatment of anxiety disorders (Zlotnick \& Huppert, 2025).
While proximal threats may involve immediate fears such as contamination or injury, core threats often reflect deeper concerns, including death, social rejection, or moral failure.
Despite their frequent application in clinical practice (commonly referred to as core fears), core threats have not been systematically studied.

Our first hypothesis addressed the discrepancy between core and proximal threats.
Previous research (Gillihan et al., 2012; Huppert \& Zlotnick, 2012; Murray, Treanor, et al., 2016) has underscored the complexity of core threats and highlighted the importance of clinicians delving deeper into patients' fears.
Consistent with these findings, we observed that proximal threats were generally distinct from their associated core threats.
Notably, only about one-third of proximal threats in the transdiagnostic samples predicted their corresponding core threats.
This suggests that identifying core threats remains a meaningful endeavor, to ensure reliable identification of core threats in the remaining two-thirds.
This distinction appears especially critical in OCD, where core threats are more diverse and proximal threats less predictive.

Contrary to our expectations, significant differences in the diversity of core versus proximal threats were observed only in the OCD sample (after applying the Holm-Bonferroni correction), and these effects were modest.
Shifting focus to the content of core threats, we observed that, across all four samples, core threats tended to concentrate on themes such as affiliation, self-image, competence, and control.
In contrast, proximal threats were often more ambiguous or focused on distress tolerance.

\subsection{Assessing Validity}\label{assessing-validity}

The development of the CTSI was driven by practical research and clinical needs.
Despite their theoretical foundation and frequent use in clinical practice, core threats lacked a validated tool for systematic measurement in research or therapy.
The CTSI was created by integrating common clinical practices with insights from the catastrophizing interview (Davey, 2006).
Early versions of the interview were manualized and refined based on expert feedback.

The current study aimed to evaluate the CTSI's validity and reliability as a tool for identifying core threats.
Face validity was established through expert clinician reviews, confirming that identified core threats aligned with those commonly observed in psychotherapy.
Construct validity was demonstrated through evidence of both convergent and divergent validity.
Participants consistently reported that their identified core threats reflected their motivations, demonstrating convergent validity.
Meanwhile, the mild associations between core and proximal threats supported divergent validity.
Furthermore, participants indicated that core threats captured their motivations better than proximal threats, highlighting their distinctiveness.

In terms of reliability, the CTSI demonstrated robust test-retest reliability, ensuring consistent identification of core threats across administrations.
It also exhibited good inter-rater reliability, with different clinicians reaching consistent conclusions.
Finally, the findings showed consistency across diverse samples, reinforcing the CTSI's applicability and relevance to various populations.

\subsection{What are We Measuring?}\label{what-are-we-measuring}

A major question remains whether fear is indeed organized around core threats to a certain extent.
This challenge has two key dimensions: theoretical and functional.

The theoretical challenge involves determining whether core threats are genuinely part of the fear structure and understanding the process by which the CTSI identifies them.
It is uncertain whether individuals possess direct verbal access to their underlying motivations.
Even if they do, questions arise about how clinicians or researchers can reliably identify these motivations.
From this perspective, core threats could be difficult to access or may not exist in the way we conceptualize them.
For instance, core threats might be generated through random processes or influenced by non-anxiety-related factors (e.g., reporting what would sound valuable or important rather than the true motivation).
Alternatively, the observed reliability over time and across interviewers could reflect participants recalling previous responses rather than genuine consistency.

We have reviewed the theoretical arguments for the existence of core threats elsewhere (Zlotnick \& Huppert, 2025).
Briefly, we propose that core threats exist in all individuals, and that the evaluations of their probability and of their meaning potentially contribute to pathological anxiety.
While not all individuals possess pathological core threats, identifying such threats in those who do can support case conceptualization and enhance the generalization of learning in psychotherapy.

Even assuming that a core threat can be determined, an interesting question arises: Is the core threat identified or construed?
Is the process of determining a core threat akin to uncovering an existing prototype (cf. Rosch \& Lloyd, 1978)?
Or should core threats be treated as ad-hoc narratives, individually tailored to activate the fear structure (cf. Barsalou, 2003)?
If the latter is the case, the therapist's role would be to help the patient construe a core threat that effectively activates the fear network as opposed to determining it.

Ultimately, these challenges may be unresolvable.
There will always be alternative explanations, as we lack direct access to these theoretical constructs (see De Houwer, 2011).
One might ask, ``If we can't be sure we can access core threats, are they scientific?''
We argue that many useful and theoretically interesting constructs are similarly inaccessible, yet remain valuable, such as attachment styles, cognitive schemas, or implicit biases.
Indirect methods, such as physiological measures of anxiety and implicit measures (Gawronski \& De Houwer, 2014), may provide access.
The accumulating data, particularly the functional implications of core threats, increasingly support the validity of this cognitive construct.

The functional challenge is as follows: Are we indeed identifying the true core threats?
How can one validate this abstract clinical construct?
One approach is to define a functional definition of core threats that \emph{can} be empirically examined (De Houwer, 2011).
Core threats primarily represent an individual's motivations to avoid certain stimuli or situations.
Indeed, preliminary findings suggested this is accurate; many individuals reported that their identified core threat indeed reflected their true motivation.
However, a more concrete functional definition is needed.

Furthermore, a significant critique of the CTSI is its potential circularity.
By defining proximal threats as distinct from core threats, participants are implicitly guided to identify a different underlying threat.
Consequently, it is unsurprising that proximal threats differ from core threats.
This issue is especially pronounced in cases involving distress-tolerance-related core threats, which are rare among core threats but common among proximal threats.
Additionally, requiring participants to articulate potential harm---even when they initially report none---may inadvertently prompt them to identify an alternative core threat.

The solution for both of these problems would lie in demonstrating that core threats, as identified through the CTSI, exhibit distinct behavioral patterns compared to proximal threats.
Core threats are believed to be more effective targets for safety learning, such as through exposure or behavioral experiments (Zlotnick \& Huppert, 2025).
Thus, demonstrating that focusing on core threats improves therapeutic outcomes would provide strong evidence for their validity.

\subsection{Distress Tolerance}\label{distress-tolerance}

In the face-to-face samples, particularly in Experiment 1, distress tolerance frequently emerged as a proximal threat but was rare as a core threat.
This suggests that distress tolerance may serve as a form of avoidance rather than an ultimate feared outcome.
Distress tolerance is often expressed in broad, universally valid terms (e.g., ``I can't handle this discomfort''), which can enable individuals to avoid confronting the specific, more threatening fears underlying their distress.
In essence, focusing on distress tolerance can provide an avenue for avoidance by shifting attention away from addressing the actual threat.
Interestingly, this pattern was not observed in the self-administered versions of the CTSI, possibly because the detailed prompts in the self-administered format encouraged participants to identify harm-avoidant threats explicitly.

Recent literature emphasizes distress tolerance as a central mechanism in anxiety disorders (Barlow et al., 2010; S. C. Hayes et al., 2006; Keough et al., 2010).
Literature on OCD highlights the importance of ``not just right'' experiences, sensory phenomena, incompleteness, and other seemingly harmless phenomena (Ecker \& Gönner, 2008; e.g., Ferrão et al., 2012).
An association between these phenomena and other anxiety disorders has also been found (Michel et al., 2016).
Indeed, distress tolerance has been shown to correlate with psychopathology (Leyro et al., 2010) and has been suggested as a treatment target in various contexts, such as smoking cessation (Brown, 2022) and borderline personality disorder (Linehan, 1993).

Distress tolerance has been proposed as a primary transdiagnostic process to target in emotional disorders (Barlow et al., 2010).
Building on this model, we emphasize harm avoidance as a complementary mechanism that warrants equal consideration.
The prevalence of distress tolerance related proximal threats, and rarity of distress tolerance related core threats suggests that, even when distress tolerance is evident, harm avoidance likely plays a significant role in anxiety.
Recent findings further support this notion, indicating that at least part of the motivation underlying NJRE in OCD arises from interference with cognitive processes rather than the enjoyment of daily life (Melli et al., 2020).
In other words, beyond the challenge of tolerating distress, there remains the critical task of disconfirming underlying threats that sustain anxiety (Craske et al., 2022).

When designing transdiagnostic interventions, it is essential to address multiple underlying processes (Hofmann \& Hayes, 2019).
While distress tolerance has received considerable attention, we argue that harm avoidance---particularly in the context of core threats---should also be considered a critical addition (A. T. Beck \& Dozois, 2011; Foa \& Kozak, 1986; Steimer, 2002).

\subsection{Affiliation Core Threats}\label{affiliation-core-threats}

Affiliation consistently appeared as a prominent value in the transdiagnostic experiments.
This could reflect an over representation of socially anxious individuals in the sample, or it may indicate that the affiliation category encompasses diverse subtypes.
Differentiating between these subtypes in future research may provide greater specificity.
Alternatively, affiliation may genuinely represent the most common core threat, as suggested by theories emphasizing its evolutionary and psychological significance (Bowlby, 1969; cf. Gilbert, 2001).

\subsection{Core Threats in OCD}\label{core-threats-in-ocd}

Much of the foundational work on core threats originates in the OCD literature (Gillihan et al., 2012; Huppert \& Zlotnick, 2012).
It is therefore reasonable to hypothesize that core threats hold particular significance for individuals with OCD.
Our findings support this hypothesis and reveal notable differences between the OCD sample and the transdiagnostic samples.
In the OCD sample, proximal threats did not predict core threats at all, a pattern not observed in the transdiagnostic groups.
Additionally, core threats in the OCD sample were significantly more diverse than proximal threats, underscoring their heterogeneity.
There was also a disproportionate prevalence of distress tolerance-related proximal threats in the OCD sample, but this pattern did not extend to core threats.
These results emphasize the critical importance of identifying and addressing core threats for individuals suffering from OCD.

The observed differences between the OCD and transdiagnostic samples may stem from the inherent diversity of OCD presentations, which manifest in various forms.
Alternatively, this distinction may reflect the fact that the OCD sample was the only validated pathological group in our study.
Future research should investigate these differences further, focusing on how core and proximal threats interact across different anxiety disorders and how these relationships may inform tailored intervention strategies.

\subsection{Limitations and Further Research}\label{limitations-and-further-research}

Core threats have been theorized to play a central role in the generalization of threat (see Zlotnick \& Huppert, 2025).
If this is indeed the case, we would expect the same core threats to motivate multiple different proximal threats.
In the current study, we found some non-significant initial evidence that one core threat can underlie multiple proximal threats.
If this finding is confirmed with further research, it would suggest that safety learning targeting core threats could be a more effective intervention for anxiety disorders.
This notion would be strengthened further by demonstrating better generalization for core threat-focused learning (Murray, Treanor, et al., 2016; see Pinciotti et al., 2021).
However, the current dataset is too small to reach definitive conclusions, and further study with a larger sample size is needed.

To investigate the agreement between threats, we needed to codify them.
As no existing typology fully suited our needs, we developed one specifically for this study.
While this typology has not yet been validated, and there is a possibility of over-dividing certain categories (e.g., safety and physical discomfort) or under-dividing others (e.g., affiliation), we believe it is sufficient for exploring the associations between different threats.
Nonetheless, further psychometric work is needed to establish a robust and validated typology for core threats.

The CTSI attempts to identify the ultimate underlying threat, but it is not clear if it succeeds.
The stopping criteria used in the CTSI are accepted in the field (see Davey, 2006).
However, there is no guarantee that the true motivation is identified.
For example, should one stop at ``I will die,'' or ask further to discover that the fear is of burning in hell?
And is burning in hell the correct stopping point, or should one delve deeper?
The instructions in the CTSI are to stop once the patient can no longer find any deeper meaning, repeat themselves, or once they start getting further from the underlying threat, as evidenced by lower levels of distress.
Ultimately, this question remains open and subjective to the discretion of the interviewer.
We argue that, despite the inherent noise in the process, the answers obtained are at least better than plain proximal threats, even if they do not reach the ``true'' core threat.

We know from several studies that core threats are deceptively diverse (Zlotnick \& Huppert, 2025).
For instance, Greenberg and colleagues (2018) examined whether the underlying fear in olfactory reference syndrome centers on embarrassing oneself or offending others.
They found that individuals possess both types of motivations, regardless of their cultural background (Western vs.~Eastern).
Moreover, about a quarter of participants reported a completely unexpected concern: whether the odor indicated a medical condition.
This highlights the importance of investigating idiosyncratic fears expressed by individuals, beyond the stereotypical fears associated with specific disorders.
However, we hypothesize that different disorders exhibit distinct patterns of core threats.
For example, safety is a prominent concern in panic disorder and OCD but appears less central in social anxiety disorder, where competence and affiliation often take precedence.
Similarly, morality plays a significant role in OCD but is less pronounced in other disorders.
Thus, variability in core threats reflects both systematic, disorder-specific tendencies and individual idiosyncrasies.
Future research should focus on mapping these associations more comprehensively.

An additional limitation of this study is the low prevalence of predictability core threats.
Our clinical experience indicates that such core threats should be more prevalent (though not quite as prevalent as other motivators).
It is likely that the focus of the CTSI on specific harm-avoidance-type outcomes may have masked such core threats.
For example, some people who fear having cancer are particularly bothered by the inherent fuzziness of the situation - that they can never know for sure.
When asked what they fear, they would focus on cancer, but that is in fact over-shooting the actual motivating core threat.
To address this problem we recommend that interviewers review the downward arrow and explicitly ask what the worst outcome would be.

\subsection{Implications for Clinical Practice and Future Research}\label{implications-for-clinical-practice-and-future-research}

The findings of this study have implications for both clinical practice and future research.
Clinically, the Core Threat Structured Interview (CTSI) proves to be a valuable tool in uncovering the underlying fears that drive anxiety disorders, facilitating the development of more effective, tailored treatment plans.
This aligns with the work of Persons (2012), who emphasized the importance of individualized case formulations in cognitive-behavioral therapy.

For future research, these findings open new avenues for exploring the mechanisms underlying the stability of core threats and their impact on treatment outcomes.
The fact that the CTSI includes a self-administered, online version that appears to be reliable and valid should allow significant further research.
We contend that one major limiting factor of studying core threats to date has been the absence of such a tool.
Investigating the interaction between core threats and other psychological constructs, such as resilience and coping strategies, could further enhance our understanding of anxiety disorders and inform more comprehensive treatment approaches.

In conclusion, this study underscores the potential critical role of core threats in anxiety disorders and provides a valid and reliable structured approach to identifying and addressing these threats in clinical practice.
By focusing on the underlying fears that drive surface threats, clinicians might be able to develop more effective interventions, ultimately improving patient outcomes.

\newpage

\subsection{Author Contributions}\label{author-contributions}

Conceptualization: E. Zlotnick, J.D. Huppert;
Methodology: E. Zlotnick;
Formal Analysis: E. Zlotnick;
Writing - Original Draft Preparation: E. Zlotnick;
Writing - Review \& Editing: E. Zlotnick, J.D. Huppert;
Supervision: J.D. Huppert.

\subsection{Conflicts of Interest}\label{conflicts-of-interest}

The authors declare that there were no conflicts of interest with respect to the authorship or the publication of this article.

\subsection{Acknowledgments}\label{acknowledgments}

We would like to thank the members of our lab for their valuable insights and contributions throughout the development of this research.
Their perspectives and suggestions have been instrumental in refining our approach and interpretations.

\subsection{Funding}\label{funding}

This work was supported by Israel Science Foundation ISF 1905/20 awarded to Jonathan Huppert, Sam and Helen Beber Chair of Clinical Psychology.

\newpage

\section{References}\label{references}

\hypertarget{refs}{}
\begin{CSLReferences}{1}{0}
\leavevmode\vadjust pre{\hypertarget{ref-abramovitchSeverityBenchmarksContemporary2020}{}}%
Abramovitch, A., Abramowitz, J. S., Riemann, B. C., \& McKay, D. (2020). Severity benchmarks and contemporary clinical norms for the Obsessive-Compulsive Inventory-Revised (OCI-R). \emph{Journal of Obsessive-Compulsive and Related Disorders}, \emph{27}, 100557. \url{https://doi.org/10.1016/j.jocrd.2020.100557}

\leavevmode\vadjust pre{\hypertarget{ref-austinGoalConstructsPsychology1996}{}}%
Austin, J. T., \& Vancouver, J. B. (1996). Goal constructs in psychology: Structure, process, and content. \emph{Psychological Bulletin}, \emph{120}(3), 338--375. \url{https://doi.org/10.1037/0033-2909.120.3.338}

\leavevmode\vadjust pre{\hypertarget{ref-barlowUnifiedProtocolTransdiagnostic2010}{}}%
Barlow, D. H., Farchione, T. J., Fairholme, C. P., Ellard, K. K., Boisseau, C. L., Allen, L. B., \& May, J. T. E. (2010). \emph{Unified Protocol for Transdiagnostic Treatment of Emotional Disorders: Therapist Guide} (1 edition). Oxford University Press.

\leavevmode\vadjust pre{\hypertarget{ref-barsalouAbstractionPerceptualSymbol2003}{}}%
Barsalou, L. W. (2003). Abstraction in perceptual symbol systems. \emph{Philosophical Transactions of the Royal Society B: Biological Sciences}, \emph{358}(1435), 1177--1187. \url{https://www.ncbi.nlm.nih.gov/pmc/articles/PMC1693222/}

\leavevmode\vadjust pre{\hypertarget{ref-beckCognitiveTherapyCurrent2011}{}}%
Beck, A. T., \& Dozois, D. J. A. (2011). Cognitive therapy: current status and future directions. \emph{Annual Review of Medicine}, \emph{62}, 397--409. \url{https://doi.org/10.1146/annurev-med-052209-100032}

\leavevmode\vadjust pre{\hypertarget{ref-beckCognitiveBehaviorTherapy2011}{}}%
Beck, J. S. (2011). \emph{Cognitive behavior therapy: Basics and beyond}. Guilford press.

\leavevmode\vadjust pre{\hypertarget{ref-borkovecWorryCognitivePhenomenon1998}{}}%
Borkovec, T. D., Ray, W. J., \& Stober, J. (1998). Worry: A cognitive phenomenon intimately linked to affective, physiological, and interpersonal behavioral processes. \emph{Cognitive Therapy and Research}, \emph{22}(6), 561--576.

\leavevmode\vadjust pre{\hypertarget{ref-boutonContextAmbiguityUnlearning2002}{}}%
Bouton, M. E. (2002). Context, ambiguity, and unlearning: sources of relapse after behavioral extinction. \emph{Biological Psychiatry}, \emph{52}(10), 976--986. \url{https://doi.org/10.1016/s0006-3223(02)01546-9}

\leavevmode\vadjust pre{\hypertarget{ref-bowlbyAttachmentLoss1969}{}}%
Bowlby, J. (1969). \emph{Attachment and Loss}. Basic Books.

\leavevmode\vadjust pre{\hypertarget{ref-brownUncertaintyAnxietyAvoidance2022}{}}%
Brown, V. (2022). \emph{Uncertainty, anxiety, avoidance, and exposure}. PsyArXiv. \url{https://doi.org/10.31234/osf.io/4zykv}

\leavevmode\vadjust pre{\hypertarget{ref-craskeOptimizingExposureTherapy2022}{}}%
Craske, M. G., Treanor, M., Zbozinek, T. D., \& Vervliet, B. (2022). Optimizing exposure therapy with an inhibitory retrieval approach and the OptEx Nexus. \emph{Behaviour Research and Therapy}, \emph{152}, 104069. \url{https://doi.org/10.1016/j.brat.2022.104069}

\leavevmode\vadjust pre{\hypertarget{ref-daveyCatastrophisingInterviewProcedure2006}{}}%
Davey, G. C. L. (2006). The Catastrophising Interview Procedure. In \emph{Worry and its Psychological Disorders} (pp. 157--176). John Wiley \& Sons, Ltd. \url{https://doi.org/10.1002/9780470713143.ch10}

\leavevmode\vadjust pre{\hypertarget{ref-davidCognitionsCognitivebehavioralPsychotherapies2006}{}}%
David, D., \& Szentagotai, A. (2006). Cognitions in cognitive-behavioral psychotherapies; toward an integrative model. \emph{Clinical Psychology Review}, \emph{26}(3), 284--298. \url{https://doi.org/10.1016/j.cpr.2005.09.003}

\leavevmode\vadjust pre{\hypertarget{ref-dehouwerWhyCognitiveApproach2011}{}}%
De Houwer, J. (2011). Why the Cognitive Approach in Psychology Would Profit From a Functional Approach and Vice Versa. \emph{Perspectives on Psychological Science}, \emph{6}(2), 202--209. \url{https://doi.org/10.1177/1745691611400238}

\leavevmode\vadjust pre{\hypertarget{ref-dugasCognitiveBehavioralTreatmentGeneralized2005}{}}%
Dugas, M., \& Koerner, N. (2005). Cognitive-Behavioral Treatment for Generalized Anxiety Disorder: Current Status and Future Directions. \emph{Journal of Cognitive Psychotherapy}, \emph{19}(1), 61--81. \href{https://insights.ovid.com}{insights.ovid.com}

\leavevmode\vadjust pre{\hypertarget{ref-dweckNeedsGoalsRepresentations2017}{}}%
Dweck, C. S. (2017). From needs to goals and representations: Foundations for a unified theory of motivation, personality, and development. \emph{Psychological Review}, \emph{124}(6), 689--719. \url{https://doi.org/10.1037/rev0000082}

\leavevmode\vadjust pre{\hypertarget{ref-eckerIncompletenessHarmAvoidance2008}{}}%
Ecker, W., \& Gönner, S. (2008). Incompleteness and harm avoidance in OCD symptom dimensions. \emph{Behaviour Research and Therapy}, \emph{46}(8), 895--904. \url{https://doi.org/10.1016/j.brat.2008.04.002}

\leavevmode\vadjust pre{\hypertarget{ref-edgingtonRandomizationTests2007}{}}%
Edgington, E., Edgington, E., \& Onghena, P. (2007). \emph{Randomization Tests} (4th ed.). Chapman and Hall/CRC. \url{https://doi.org/10.1201/9781420011814}

\leavevmode\vadjust pre{\hypertarget{ref-ernstPermutationMethodsBasis2004}{}}%
Ernst, M. D. (2004). Permutation Methods: A Basis for Exact Inference. \emph{Statistical Science}, \emph{19}(4), 676--685. \url{https://doi.org/10.1214/088342304000000396}

\leavevmode\vadjust pre{\hypertarget{ref-ferraoSensoryPhenomenaAssociated2012}{}}%
Ferrão, Y. A., Shavitt, R. G., Prado, H., Fontenelle, L. F., Malavazzi, D. M., de Mathis, M. A., Hounie, A. G., Miguel, E. C., \& do Rosário, M. C. (2012). Sensory phenomena associated with repetitive behaviors in obsessive-compulsive disorder: an exploratory study of 1001 patients. \emph{Psychiatry Research}, \emph{197}(3), 253--258. \url{https://doi.org/10.1016/j.psychres.2011.09.017}

\leavevmode\vadjust pre{\hypertarget{ref-foaObsessiveCompulsiveInventoryDevelopment2002}{}}%
Foa, E. B., Huppert, J. D., Leiberg, S., Langner, R., Kichic, R., Hajcak, G., \& Salkovskis, P. M. (2002). The Obsessive-Compulsive Inventory: development and validation of a short version. \emph{Psychological Assessment}, \emph{14}(4), 485--496.

\leavevmode\vadjust pre{\hypertarget{ref-foaEmotionalProcessingFear1986}{}}%
Foa, E. B., \& Kozak, M. J. (1986). Emotional processing of fear: Exposure to corrective information. \emph{Psychological Bulletin}, \emph{99}(1), 20--35. \url{https://doi.org/10.1037/0033-2909.99.1.20}

\leavevmode\vadjust pre{\hypertarget{ref-gawronskiImplicitMeasuresSocial2014}{}}%
Gawronski, B., \& De Houwer, J. (2014). Implicit Measures in Social and Personality Psychology. In C. M. Judd \& H. T. Reis (Eds.), \emph{Handbook of Research Methods in Social and Personality Psychology} (2nd ed., pp. 283--310). Cambridge University Press. \url{https://doi.org/10.1017/CBO9780511996481.016}

\leavevmode\vadjust pre{\hypertarget{ref-gilbertEVOLUTIONSOCIALANXIETY2001}{}}%
Gilbert, P. (2001). EVOLUTION AND SOCIAL ANXIETY: The Role of Attraction, Social Competition, and Social Hierarchies. \emph{Psychiatric Clinics}, \emph{24}(4), 723--751. \url{https://doi.org/10.1016/S0193-953X(05)70260-4}

\leavevmode\vadjust pre{\hypertarget{ref-gillihanCommonPitfallsExposure2012}{}}%
Gillihan, S. J., Williams, M. T., Malcoun, E., Yadin, E., \& Foa, E. B. (2012). Common pitfalls in exposure and response prevention (EX/RP) for OCD. \emph{Journal of Obsessive-Compulsive and Related Disorders}, \emph{1}(4), 251--257. \url{https://doi.org/10.1016/j.jocrd.2012.05.002}

\leavevmode\vadjust pre{\hypertarget{ref-greenbergEgocentricAllocentricFears2018}{}}%
Greenberg, J. L., Weingarden, H., \& Wilhelm, S. (2018). Egocentric and allocentric fears in olfactory reference syndrome. \emph{Journal of Obsessive-Compulsive and Related Disorders}, \emph{16}, 72--75. \url{https://doi.org/10.1016/j.jocrd.2018.01.001}

\leavevmode\vadjust pre{\hypertarget{ref-grosPsychometricPropertiesStateTrait2007}{}}%
Grös, D. F., Antony, M. M., Simms, L. J., \& McCabe, R. E. (2007). Psychometric properties of the State-Trait Inventory for Cognitive and Somatic Anxiety (STICSA): Comparison to the State-Trait Anxiety Inventory (STAI). \emph{Psychological Assessment}, \emph{19}, 369--381. \url{https://doi.org/10.1037/1040-3590.19.4.369}

\leavevmode\vadjust pre{\hypertarget{ref-hallgrenComputingInterRaterReliability2012}{}}%
Hallgren, K. A. (2012). Computing Inter-Rater Reliability for Observational Data: An Overview and Tutorial. \emph{Tutorials in Quantitative Methods for Psychology}, \emph{8}(1), 23--34. \url{https://www.ncbi.nlm.nih.gov/pmc/articles/PMC3402032/}

\leavevmode\vadjust pre{\hypertarget{ref-harveyCatastrophicWorryPrimary2003}{}}%
Harvey, A. G., \& Greenall, E. (2003). Catastrophic worry in primary insomnia. \emph{Journal of Behavior Therapy and Experimental Psychiatry}, \emph{34}(1), 11--23. \url{https://doi.org/10.1016/s0005-7916(03)00003-x}

\leavevmode\vadjust pre{\hypertarget{ref-hayesAnsweringCallStandard2007}{}}%
Hayes, A. F., \& Krippendorff, K. (2007). Answering the Call for a Standard Reliability Measure for Coding Data. \emph{Communication Methods and Measures}, \emph{1}(1), 77--89. \url{https://doi.org/10.1080/19312450709336664}

\leavevmode\vadjust pre{\hypertarget{ref-hayesAcceptanceCommitmentTherapy2006}{}}%
Hayes, S. C., Luoma, J. B., Bond, F. W., Masuda, A., \& Lillis, J. (2006). Acceptance and commitment therapy: Model, processes and outcomes. \emph{Behaviour Research and Therapy}, \emph{44}(1), 1--25.

\leavevmode\vadjust pre{\hypertarget{ref-hofmannFutureInterventionScience2019}{}}%
Hofmann, S. G., \& Hayes, S. C. (2019). The Future of Intervention Science: Process-Based Therapy. \emph{Clinical Psychological Science}, \emph{7}(1), 37--50. \url{https://doi.org/10.1177/2167702618772296}

\leavevmode\vadjust pre{\hypertarget{ref-holmSimpleSequentiallyRejective1979}{}}%
Holm, S. (1979). A Simple Sequentially Rejective Multiple Test Procedure. \emph{Scandinavian Journal of Statistics}, \emph{6}(2), 65--70. \url{https://www.jstor.org/stable/4615733}

\leavevmode\vadjust pre{\hypertarget{ref-holmesMentalImageryEmotion2005}{}}%
Holmes, E. A., \& Mathews, A. (2005). Mental Imagery and Emotion: A Special Relationship? \emph{Emotion}, \emph{5}(4), 489--497. \url{https://doi.org/10.1037/1528-3542.5.4.489}

\leavevmode\vadjust pre{\hypertarget{ref-holmesMentalImageryEmotion2010}{}}%
Holmes, E. A., \& Mathews, A. (2010). Mental imagery in emotion and emotional disorders. \emph{Clinical Psychology Review}, \emph{30}(3), 349--362. \url{https://doi.org/10.1016/j.cpr.2010.01.001}

\leavevmode\vadjust pre{\hypertarget{ref-huppertOCIRValidationSubscales2007}{}}%
Huppert, J. D., Walther, M. R., Hajcak, G., Yadin, E., Foa, E. B., Simpson, H. B., \& Liebowitz, M. R. (2007). The OCI-R: validation of the subscales in a clinical sample. \emph{Journal of Anxiety Disorders}, \emph{21}(3), 394--406. \url{https://doi.org/10.1016/j.janxdis.2006.05.006}

\leavevmode\vadjust pre{\hypertarget{ref-huppertCoreFearsValues2012}{}}%
Huppert, J. D., \& Zlotnick, E. (2012). Core fears, values, and obsessive-compulsive disorder: a preliminary clinical-theoretical outlook. \emph{Psicoterapia Cognitiva e Comportamentale}, \emph{18}(1), 91--102.

\leavevmode\vadjust pre{\hypertarget{ref-kaderVariabilityCategoricalVariables2007}{}}%
Kader, G. D., \& Perry, M. (2007). Variability for Categorical Variables. \emph{Journal of Statistics Education}, \emph{15}(2), null. \url{https://doi.org/10.1080/10691898.2007.11889465}

\leavevmode\vadjust pre{\hypertarget{ref-kendallFutureCognitiveAssessment1987}{}}%
Kendall, P. C., \& Ingram, R. (1987). The future for cognitive assessment of anxiety: Let's get specific. In M. Larry \& L. M. Ascher (Eds.), \emph{Anxiety and stress disorders: Cognitive-behavioral assessment and treatment} (pp. pp. 89--104). Guilford Press.

\leavevmode\vadjust pre{\hypertarget{ref-keoughAnxietySymptomatologyAssociation2010}{}}%
Keough, M. E., Riccardi, C. J., Timpano, K. R., Mitchell, M. A., \& Schmidt, N. B. (2010). Anxiety Symptomatology: The Association With Distress Tolerance and Anxiety Sensitivity. \emph{Behavior Therapy}, \emph{41}(4), 567--574. \url{https://doi.org/10.1016/j.beth.2010.04.002}

\leavevmode\vadjust pre{\hypertarget{ref-kroenkePHQ8MeasureCurrent2009}{}}%
Kroenke, K., Strine, T. W., Spitzer, R. L., Williams, J. B. W., Berry, J. T., \& Mokdad, A. H. (2009). The PHQ-8 as a measure of current depression in the general population. \emph{Journal of Affective Disorders}, \emph{114}(1), 163--173. \url{https://doi.org/10.1016/j.jad.2008.06.026}

\leavevmode\vadjust pre{\hypertarget{ref-leahyCognitiveTherapyTechniques2003}{}}%
Leahy, R. L. (2003). \emph{Cognitive therapy techniques: a practitioner's guide}. Guilford Press.

\leavevmode\vadjust pre{\hypertarget{ref-leyroDistressTolerancePsychopathological2010}{}}%
Leyro, T. M., Zvolensky, M. J., \& Bernstein, A. (2010). Distress Tolerance and Psychopathological Symptoms and Disorders: A Review of the Empirical Literature among Adults. \emph{Psychological Bulletin}, \emph{136}(4), 576--600. \url{https://doi.org/10.1037/a0019712}

\leavevmode\vadjust pre{\hypertarget{ref-linehanCognitivebehavioralTreatmentBorderline1993}{}}%
Linehan, M. M. (1993). \emph{Cognitive-behavioral treatment of borderline personality disorder} (pp. xvii, 558). Guilford Press.

\leavevmode\vadjust pre{\hypertarget{ref-melliAssessingBeliefsConsequences2020}{}}%
Melli, G., Moulding, R., Puccetti, C., Pinto, A., Caccico, L., Drabik, M. J., \& Sica, C. (2020). Assessing beliefs about the consequences of not just right experiences: Psychometric properties of the Not Just Right Experience-Sensitivity Scale (NJRE-SS). \emph{Clinical Psychology \& Psychotherapy}, \emph{27}(6), 847--857. \url{https://doi.org/10.1002/cpp.2468}

\leavevmode\vadjust pre{\hypertarget{ref-michelEmotionalDistressTolerance2016}{}}%
Michel, N. M., Rowa, K., Young, L., \& McCabe, R. E. (2016). Emotional distress tolerance across anxiety disorders. \emph{Journal of Anxiety Disorders}, \emph{40}, 94--103. \url{https://doi.org/10.1016/j.janxdis.2016.04.009}

\leavevmode\vadjust pre{\hypertarget{ref-moorePsychometricEvaluationOverall2015}{}}%
Moore, S. A., Welch, S. S., Michonski, J., Poquiz, J., Osborne, T. L., Sayrs, J., \& Spanos, A. (2015). Psychometric evaluation of the Overall Anxiety Severity And Impairment Scale (OASIS) in individuals seeking outpatient specialty treatment for anxiety-related disorders. \emph{Journal of Affective Disorders}, \emph{175}, 463--470. \url{https://doi.org/10.1016/j.jad.2015.01.041}

\leavevmode\vadjust pre{\hypertarget{ref-mundtWorkSocialAdjustment2002}{}}%
Mundt, J. C., Marks, I. M., Shear, M. K., \& Greist, J. M. (2002). The Work and Social Adjustment Scale: a simple measure of impairment in functioning. \emph{The British Journal of Psychiatry}, \emph{180}(5), 461--464. \url{https://doi.org/10.1192/bjp.180.5.461}

\leavevmode\vadjust pre{\hypertarget{ref-murrayDissectingCoreFear2016}{}}%
Murray, S. B., Loeb, K. L., \& Grange, D. L. (2016). Dissecting the Core Fear in Anorexia Nervosa: Can We Optimize Treatment Mechanisms? \emph{JAMA Psychiatry}, \emph{73}(9), 891--892. \url{https://doi.org/10.1001/jamapsychiatry.2016.1623}

\leavevmode\vadjust pre{\hypertarget{ref-murrayExtinctionTheoryAnorexia2016}{}}%
Murray, S. B., Treanor, M., Liao, B., Loeb, K. L., Griffiths, S., \& Le Grange, D. (2016). Extinction theory \& anorexia nervosa: Deepening therapeutic mechanisms. \emph{Behaviour Research and Therapy}, \emph{87}, 1--10. \url{https://doi.org/10.1016/j.brat.2016.08.017}

\leavevmode\vadjust pre{\hypertarget{ref-normanDevelopmentValidationOverall2006}{}}%
Norman, S. B., Hami Cissell, S., Means-Christensen, A. J., \& Stein, M. B. (2006). Development and validation of an Overall Anxiety Severity And Impairment Scale (OASIS). \emph{Depression and Anxiety}, \emph{23}(4), 245--249. \url{https://doi.org/10.1002/da.20182}

\leavevmode\vadjust pre{\hypertarget{ref-padeskySocraticQuestioningChanging1993}{}}%
Padesky, C. A. (1993). Socratic questioning: Changing minds or guiding discovery. \emph{A Keynote Address Delivered at the European Congress of Behavioural and Cognitive Therapies, London}, \emph{24}.

\leavevmode\vadjust pre{\hypertarget{ref-personsCaseFormulationApproach2012}{}}%
Persons, J. B. (2012). \emph{The Case Formulation Approach to Cognitive-Behavior Therapy} (Reprint edition). The Guilford Press.

\leavevmode\vadjust pre{\hypertarget{ref-pinciottiCallActionRecommendations2021}{}}%
Pinciotti, C. M., Smith, Z., Singh, S., Wetterneck, C. T., \& Williams, M. T. (2021). Call to action: Recommendations for justice-based treatment of obsessive-compulsive disorder with sexual orientation and gender themes. \emph{Behavior Therapy}. \url{https://doi.org/10.1016/j.beth.2021.11.001}

\leavevmode\vadjust pre{\hypertarget{ref-reeDistinguishingCognitiveSomatic2008}{}}%
Ree, M. J., French, D., MacLeod, C., \& Locke, V. (2008). Distinguishing Cognitive and Somatic Dimensions of State and Trait Anxiety: Development and Validation of the State-Trait Inventory for Cognitive and Somatic Anxiety (STICSA). \emph{Behavioural and Cognitive Psychotherapy}, \emph{36}(3), 313--332. \url{https://doi.org/10.1017/S1352465808004232}

\leavevmode\vadjust pre{\hypertarget{ref-roschCognitionCategorization1978}{}}%
Rosch, E., \& Lloyd, B. B. (1978). \emph{Cognition and categorization}.

\leavevmode\vadjust pre{\hypertarget{ref-ryanOxfordHandbookHuman2012}{}}%
Ryan, R. M. (2012). \emph{The Oxford Handbook of Human Motivation}. Oxford University Press, USA.

\leavevmode\vadjust pre{\hypertarget{ref-safranElicitingHotCognitions1982}{}}%
Safran, J. D., \& Greenberg, L. S. (1982). Eliciting {``hot cognitions''} in cognitive behaviour therapy: Rationale and procedural guidelines. \emph{Canadian Psychology/Psychologie Canadienne}, \emph{23}(2), 83--87. \url{https://doi.org/10.1037/h0081247}

\leavevmode\vadjust pre{\hypertarget{ref-samuelSurveyInterviewMethods2020}{}}%
Samuel, D. B., Bucher, M. A., \& Suzuki, T. (2020). Survey and Interview Methods. In A. G. C. Wright \& M. N. Hallquist (Eds.), \emph{The Cambridge Handbook of Research Methods in Clinical Psychology} (pp. 45--53). Cambridge University Press. \url{https://doi.org/10.1017/9781316995808.007}

\leavevmode\vadjust pre{\hypertarget{ref-schwartzUniversalsContentStructure1992}{}}%
Schwartz, S. H. (1992). Universals in the content and structure of values: Theoretical advances and empirical tests in 20 countries. \emph{Advances in Experimental Social Psychology}, \emph{25}(1), 1--65.

\leavevmode\vadjust pre{\hypertarget{ref-simpsonMeasurementDiversity1949}{}}%
Simpson, E. H. (1949). Measurement of Diversity. \emph{Nature}, \emph{163}(4148), 688--688. \url{https://doi.org/10.1038/163688a0}

\leavevmode\vadjust pre{\hypertarget{ref-steimerBiologyFearAnxietyrelated2002}{}}%
Steimer, T. (2002). The biology of fear- and anxiety-related behaviors. \emph{Dialogues in Clinical Neuroscience}, \emph{4}(3), 231--249. \url{https://www.ncbi.nlm.nih.gov/pmc/articles/PMC3181681/}

\leavevmode\vadjust pre{\hypertarget{ref-tolinPsychometricPropertiesStructured2018}{}}%
Tolin, D. F., Gilliam, C., Wootton, B. M., Bowe, W., Bragdon, L. B., Davis, E., Hannan, S. E., Steinman, S. A., Worden, B., \& Hallion, L. S. (2018). Psychometric Properties of a Structured Diagnostic Interview for DSM-5 Anxiety, Mood, and Obsessive-Compulsive and Related Disorders. \emph{Assessment}, \emph{25}(1), 3--13. \url{https://doi.org/10.1177/1073191116638410}

\leavevmode\vadjust pre{\hypertarget{ref-vaseyCatastrophizingAssessmentWorrisome1992}{}}%
Vasey, M. W., \& Borkovec, T. D. (1992). A catastrophizing assessment of worrisome thoughts. \emph{Cognitive Therapy and Research}, \emph{16}(5), 505--520. \url{https://doi.org/10.1007/BF01175138}

\leavevmode\vadjust pre{\hypertarget{ref-watkinsMoodInputRumination2002}{}}%
Watkins, E., \& Mason, A. (2002). Mood as input and rumination. \emph{Personality and Individual Differences}, \emph{32}(4), 577--587. \url{https://doi.org/10.1016/S0191-8869(01)00058-7}

\leavevmode\vadjust pre{\hypertarget{ref-zlotnickAnatomyFearCloser2025}{}}%
Zlotnick, E., \& Huppert, J. D. (2025). \emph{The Anatomy of Fear: A Closer Look at Core Threats} {[}Paper submitted for publication{]}.

\leavevmode\vadjust pre{\hypertarget{ref-zuromskiDevelopingOptimalShortform2019}{}}%
Zuromski, K. L., Ustun, B., Hwang, I., Keane, T. M., Marx, B. P., Stein, M. B., Ursano, R. J., \& Kessler, R. C. (2019). Developing an optimal short-form of the PTSD Checklist for DSM-5 (PCL-5). \emph{Depression and Anxiety}, \emph{36}(9), 790--800. \url{https://doi.org/10.1002/da.22942}

\end{CSLReferences}


\end{document}
